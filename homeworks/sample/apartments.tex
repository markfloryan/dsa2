\documentclass[11pt]{article}
%\usepackage{fancyheadings}
\usepackage{wrapfig}
\usepackage{epsfig}
\usepackage{hyperref}
\setlength{\headheight}{0pt}
%\setlength{\footheight}{0pt}
\setlength{\topmargin}{-.5in}
\setlength{\oddsidemargin}{-0.25in}
\setlength{\evensidemargin}{-0.25in}
\setlength{\textwidth}{7truein}
\setlength{\textheight}{9truein}
\setlength{\parskip}{6pt}

\begin{document}

\section*{Hunting for an Apartment}

%\subsection*{Description}

\begin{wrapfigure}{r}{3in}
\vspace{-10pt}
\epsfig{figure=park,width=3in}
\vspace{-30pt}
\end{wrapfigure}

Professor Horton is looking for a new apartment. He wants to live somewhere affordable, but he also doesn't want to live in the absolute cheapest apartment he can find. Given a list of monthly rents for apartments in his area, can you compute the second cheapest apartment that he can rent?

\subsection*{Input}
The input file will begin with one line containing $2 \leq n \leq 20$, the number of apartments available. The following $n$ lines will each contain a single integer $l_i \leq 1000$ describing monthly price of apartment $i$.

\subsection*{Output}
Output the second cheapest apartment price of the ones provided.



\vspace{0.25in}\hspace{-0.3in}\begin{tabular}{ll}

%\subsection*{Sample Input}
\parbox{3in}{{\large\bf Sample Input}

\vspace{0.15in}

{\tt 
5\linebreak
100\linebreak
150\linebreak
90\linebreak
80\linebreak
3500
}
}

&

\parbox{3in}{{\large\bf Sample Output}

\vspace{0.15in}

{\tt
90
}
}

\\
\end{tabular}

\end{document}
